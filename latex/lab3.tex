\documentclass[oneside,final,14pt]{extreport}
\usepackage[utf8]{inputenc}
\usepackage[T2A]{fontenc}
\usepackage[russian]{babel}
\usepackage{vmargin}
\setpapersize{A4}
\setmarginsrb{2cm}{1.5cm}{1cm}{1.5cm}{0pt}{0mm}{0pt}{13mm}
\usepackage{indentfirst}
\sloppy
\setcounter{secnumdepth}{3}
\setcounter{tocdepth}{3}
\usepackage{color}
\usepackage{graphicx}
\usepackage{amsmath}
\usepackage{animate}
\usepackage{hyperref}
\usepackage{underscore}
\usepackage{listings}
\lstset{
    language = python, 
    frame=single, 
    captionpos = b,
    numbers = left,
    basicstyle=\footnotesize
    }
    
\newcommand\marker[2][green]{{\fboxsep=0pt\colorbox{#1}{\strut #2}}}
\newcommand\chap[1]{
    \section*{#1}
    \addcontentsline{toc}{chapter}{#1}
}


\begin{document}

\begin{titlepage}
    \begin{center}
        \large
        МИНИСТЕРСТВО ОБРАЗОВАНИЯ И НАУКИ РОССИЙСКОЙ ФЕДЕРАЦИИ
        
        Федеральное агентство по образованию

        "<Пермский национальный исследовательский политехнический университет">
        
        Кафедра микропроцессорных средств автоматизации
        \vfill

        Отчёт по лабораторной работе \textnumero 3
        
        По дисциплине компьютерная графика
        
        На тему: "<Алгоритмы удаления невидимых линий">
      
    \end{center}
    \vfill
   
    \newlength{\ML}
    \settowidth{\ML}{«\underline{\hspace{0.7cm}}» \underline{\hspace{1.5cm}}}
    \hfill
    \begin{minipage}{0.5\textwidth}
        Работу выполнили\\
        студенты гр. ИСУП-18-2м\\
        \underline{\hspace{\ML}} А.\,С.~Морозов\\
        \underline{\hspace{\ML}} В.\,О.~Раскошинский\\
    \end{minipage}%
    \bigskip
   
    \hfill\begin{minipage}{0.5\textwidth}
        Проверил доцент кафедры МСА\\
        \underline{\hspace{\ML}} Л.\,А.~Мыльников\\
    \end{minipage}%
    \vfill
   
    \begin{center}
        Пермь, 2020 г.
    \end{center}
\end{titlepage}

\newpage

\chap{Описание алгоритма}

Наиболее часто алгоритм плавающего горизонта применяется для удаления невидимых линий при отображении функций, описывающих поверхность в виде $f(x,y,z)=0$. Подобные функции возникают во многих приложениях в математике, технике, естественных науках и других дисциплинах.

Главная идея данного метода заключается в сведении трехмерной задачи к двумерной путем пересечения исходной поверхности последовательностью параллельных секущих плоскостей, имеющих постоянные значения координат $x$, $y$ или $z$.

Параллельные плоскости определяются постоянными значениями $z$. Функция $f(x,y,z)=0$ сводится к последовательности кривых, лежащих в каждой из этих параллельных плоскостей, например к последовательности $y=f(x,z)$ или ${x=g(y,z)}$, где $z$ постоянно на каждой из заданных параллельных плоскостей.

Алгоритм следующий: если на текущей плоскости при некотором заданном значении $x$ соответствующее значение $y$ на кривой больше максимума или меньше минимума по $y$ для всех предыдущих кривых при этом значении $x$, то текущая кривая видима. В противном случае она невидима.

\chap{Реализация алгоритма}

Файл \texttt{subroutine.py} содержит функцию \texttt{horizon}, которая реализует алгоритм плавающего горизонта, представлена в листинге \ref{sub:horizon}.

\begin{lstlisting}[caption = Функция \texttt{horizon} -- реализация алгоритма плавающего горизонта, label = sub:horizon]
def horizon(x1, y1, x2, y2, top, bottom, image):
    x = x1
    y = y1
    dx = x2 - x1
    dy = y2 - y1
    sx = sign(dx)
    sy = sign(dy)
    dx = abs(dx)
    dy = abs(dy)

    if dx == 0 and dy == 0 and 0 <= x < image.width():
        if y >= top[x]:
            top[x] = y
            image.setPixel(x, image.height() - y, Qt.white)

        if y <= bottom[x]:
            bottom[x] = y
            image.setPixel(x, image.height() - y, Qt.white)

        return top, bottom

    change = 0
    if dy > dx:
        dx, dy = dy, dx
        change = 1

    y_max_curr = top[x]
    y_min_curr = bottom[x]
    e = 2 * dy - dx

    i = 1
    while i <= dx:
        if 0 <= x < image.width():

            if y >= top[x]:
                if y >= y_max_curr:
                    y_max_curr = y
                image.setPixel(x, image.height() - y, Qt.white)

            if y <= bottom[x]:
                if y <= y_min_curr:
                    y_min_curr = y
                image.setPixel(x, image.height() - y, Qt.white)

        if e >= 0:
            if change:
                top[x] = y_max_curr
                bottom[x] = y_min_curr

                x += sx

                y_max_curr = top[x]
                y_min_curr = bottom[x]

            else:
                y += sy

            e -= 2 * dx
        if e < 0:
            if not change:
                top[x] = y_max_curr
                bottom[x] = y_min_curr

                x += sx

                y_max_curr = top[x]
                y_min_curr = bottom[x]

            else:
                y += sy

            e += 2 * dy

        i += 1

    return top, bottom
\end{lstlisting}

Данная функция импортируется в файл \texttt{horizon.py} и используется совместно с функциями поворота координат \texttt{rotateX}, \texttt{rotateY}, \texttt{rotateZ} и \texttt{tranform} в функции \verb:float_horizon:. Исходный код функции \verb:float_horizon: представлен в листинге \ref{horizon:float_horizon}.

\vfill

\begin{lstlisting}[caption = {Функция \texttt{float_horizon} -- использование алгоритма}, label = horizon:float_horizon]
def float_horizon(scene_width, scene_hight, x_min, x_max, 
                  x_step, z_min, z_max, z_step,
                  tx, ty, tz, func, image):
    x_right = -1
    y_right = -1
    x_left = -1
    y_left = -1

    # initialization of horizon arrays
    
    z = z_max
    while z >= z_min:
        z_buf = z
        x_prev = x_min
        y_prev = func(x_min, z)
        x_prev, y_prev, z_buf = tranform(x_prev, y_prev, z, tx, ty, tz)

        # process the left edge (look at the previous with the current)
        if x_left != -1:
            top, bottom = horizon(x_prev, y_prev, x_left, y_left, top, bottom, image)
        x_left = x_prev
        y_left = y_prev

        x = x_min
        while x <= x_max:
            y = func(x, z)
            x_curr, y_curr, z_buf = tranform(x, y, z, tx, ty, tz)

            # Adding to the horizon and drawing lines
            # start to draw not from the previous one 
            # but already from the transformed
            top, bottom = horizon(x_prev, y_prev, x_curr, y_curr, top, bottom, image)
            x_prev = x_curr
            y_prev = y_curr

            x += x_step

        # process the right edge (look at the current one with the following)
        if z != z_max:
            x_right = x_max
            y_right = func(x_max, z - z_step)
            x_right, y_right, z_buf = tranform(x_right, y_right, 
                                               z-z_step, tx, ty, tz)
            top, bottom = horizon(x_prev, y_prev, x_right, y_right, 
                                  top, bottom, image)

        z -= z_step

    return image
\end{lstlisting}

В то же время функция \texttt{float_horizon} импортируется в файл \texttt{lab.py} и используется в функции \texttt{draw}. Исходный код функции \texttt{draw} представлен в листинге \ref{lab:draw}. Функция \texttt{draw} используется в классе \texttt{Window} для визуализации результатов при изменении входных значений, полученных из пользовательского интерфейса, который описан в файле \texttt{window.ui}. 

\begin{lstlisting}[caption = Функция \texttt{draw} -- визуализация результатов, label = lab:draw]
def draw(win):
    win.scene.clear()
    win.image.fill(black)
    tx = win.dial_x.value()
    ty = win.dial_y.value()
    tz = win.dial_z.value()

    if win.funcs.currentText() == "- sin(x**2 + y**2)":
        f = f1

    if win.funcs.currentText() == "(sin(sqrt(x**2 + y**2)))/(sqrt(x**2 + y**2))":
        f = f2

    if win.funcs.currentText() == "exp((- x**2 - y**2) * x)":
        f = f3

    if win.funcs.currentText() == "sqrt(x**2+y**2)+3*(cos(sqrt(x**2+y**2)))+5":
        f = f4

    win.image = float_horizon(win.scene.width(), 
                              win.scene.height(), 
                              win.x_min.value(), 
                              win.x_max.value(), 
                              win.dx.value(),
                              win.z_min.value(), 
                              win.z_max.value(), 
                              win.dz.value(), 
                              tx, ty, tz, f, 
                              win.image)

    pix = QPixmap()
    pix.convertFromImage(win.image)
    win.scene.addPixmap(pix)
\end{lstlisting}

Графический интерфейс пользователя представлен на рисунках \ref{gui:1}, \ref{gui:2} и \ref{gui:3}.

Весь код и результаты его работы представлены в открытом \href{https://github.com/morozov6420/floating_horizon_algorithm}{репозитории}.

\begin{figure}[ht]
    \minipage{0.4\textwidth}
    \includegraphics[width=\linewidth]{"../screens/f1\string_1.png"}
    \endminipage\hfill
    \minipage{0.4\textwidth}
    \includegraphics[width=\linewidth]{"../screens/f1\string_2.png"}
    \endminipage\hfill
    \caption{\(z=-sin\left(x^2 + y^2\right)\)}
    \label{gui:1}
\end{figure}

\begin{figure}[ht]
    \minipage{0.5\textwidth}
    \includegraphics[width=\linewidth]{"../screens/f2\string_1.png"}
    \endminipage\hfill
    \minipage{0.5\textwidth}
    \includegraphics[width=\linewidth]{"../screens/f2\string_2.png"}
    \endminipage\hfill
    \caption{\(z=\frac{sin\left(\sqrt{x^2 + y^2}\right)}{\sqrt{x^2 + y^2}}\)}
    \label{gui:2}
\end{figure}

\begin{figure}[ht]
    \minipage{0.5\textwidth}
    \includegraphics[width=\linewidth]{"../screens/f4\string_1.png"}
    \endminipage\hfill
    \minipage{0.5\textwidth}
    \includegraphics[width=\linewidth]{"../screens/f4\string_2.png"}
    \endminipage\hfill
    \caption{\(z=\sqrt{x^2 + y^2} + 3 \cdot cos\left(\sqrt{x^2 + y^2}\right)+5\)}
    \label{gui:3}
\end{figure}

\end{document}